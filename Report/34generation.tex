In the implementation the generation of source code is done
through a model to text transformation. The generation outputs compilable
Java based on input from an ECDAR file.

The Eclipse Modeling Framework (EMF) is utilized for the process. EMF is a
modeling framework and code generation facility for building applications based
on a structured data model. From a model specification described in the XMI-format, EMF
provides tools and run time support to produce a set of Java classes for the
model, along with a set of adapter classes that enable viewing and command-based
editing of the model and it provides a basic editor. The core EMF framework
includes a meta model -- in Ecore -- for describing models and run time support for the
models including: change notification, persistence support with default XMI
serialization, and a very efficient reflective API for manipulating EMF objects
generically. In the implementation presented in this paper an Xtext environment
is generated from the ECDAR Ecore model. Xtext is a framework for development of
programming languages and domain specific languages.

In order to generate code from the model it's imperative to follow a process of
multiple steps: Get the input from ECDAR, translate this to Xtext ECDAR DSL,
setup a workflow that manages the process and finally a Xpand-template is needed
to define how the transformation output should look like. Each step is described in more detail in the following section.

\subsubsection{Transformation Process}
\label{transformation-process}

The initial output from the ECDAR tool is in XML-format. The XML-output
contains a complete definition of the model with locations, edges, variables,
transformations etc. In order to work with these files and do the actual code
generation, a conversion to Xtext ECDAR DSL is needed. For this conversion we
are using a converter (courtesy of Bastian M\"uller) that simply takes the ECDAR XML-file and converts it to ECDAR DSL. The ECDAR DSL syntax is defined in our Xtext ECDAR
environment. With the combination of the Ecore meta model and the Xtext syntax a
workflow can be defined. This workflow is describing how to handle the
generation process. This is done with the help of the ``Modeling Workflow Engine 2''
(MWE2). Also referenced in the workflow is the template that describes how the
actual output is going to look like. The templates are written using
Xpand. Xpand is a statically typed template language. Conveniently Xpand
supports code-completion directly connected to the Ecore model defined in the
MWE2 workflow, but also comes with syntax coloring, refactoring and error
highlighting. The output generation results, for the system presented in this
paper, are based on several workflows and templates to do the rather complex
transformations: One set of workflows and templates for respectively the
Specification, Controller and Environment.

More specifically in the workflow-file one defines what model to use, a
slot-name to refer to later and an entry point. The entry point defines which
class element is the top or root element. The entry-element that is specified for
the three aforementioned workflows is ``ETSpecificationDefinition''. Also defined
in the workflow is how to use the entry-element. For instance in the
specification workflow it is defined that for each "ETSpecificationDefinition" a
transformation is done using the Xpand template for this particular
generation. The end result is generated output for each specification that
was initially described and modeled from within the ECDAR XML-file.

With the workflow fully configured, the next step is to write the
transformation. This is done in a Xpand template. The snippets in Fig.
\ref{xpand-example} shows some important steps.

\begin{figure}[t]
\lstinputlisting[linerange={1-2}]{code/TemplateSpecifications.xpt}
\lstinputlisting[linerange={6-6}]{code/TemplateSpecifications.xpt}
\lstinputlisting[linerange={34-41}]{code/TemplateSpecifications.xpt}
\caption{Snippets from Xpand-template \label{xpand-example}}
\end{figure}

The arrows, known as guillemots (``\guillemotleft'' and ``\guillemotright''),
indicates where in figure \ref{xpand-example} the XPAND language is in place. First
of all an import of the model is done in the first line, referenced as
ecdarText. We then proceed to one of the central concepts of Xpand by using the
define-block; This is where we define our template. We only use one template in
this specific file, but it could have contained multiple, which would have
resulted in multiple define-blocks. In the next and last snippet we jump to a
part where we are iterating through each edge and create a constructor for the
current class. In the first line we create the constructor by inserting the text
"Edge" and add the number the iterator has reached. We then iterate through a
list of the current Edges variables, which should be one signal, and returns the
results as a list. We furthermore use a new iterator to keep track of this
iteration. Afterward we use a check too make sure it's the first iteration, and if it is, we
print out the current Edge target name and the variable signal, such as \texttt{super(C,
"signal");}.

The notion of tasks as previous described in Sec.~\ref{subsec:tasks} is accounted for in
our generated output. In the controller we generate functions that will be
invoked at each location. A task is a procedure which will be executed as soon
as an automaton traverses over an edge, arriving at a new location. There can
only be one task for each location. The idea behind having all methods in the
controller is for a better overview and a centralized customizable file.
