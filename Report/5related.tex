ECDAR is a TIOA modeler based on UPPAAL. A similar tool based on UPPAAL already
exists.

\paragraph{Times}
The aim of this project is to develop a tool for automatic synthesis of Java-code from the modelling tool ECDAR. However there already exist tools similar too ecdar that provide code-generation.

Simulor to tools based on timed automata exists in relation to ECDAR. Most noticeable TIMES\footnote{\url{http://www.timestool.com}}: is a graphical tool set for modelling, schedulability analysis for implementation of embedded systems. It allows users to model a system and the abstract behaviour of its environment. 
Like ECDAR, TIMES is based on timed automata (See section x for more details on automaton). Similarly TIMES derive from UPPAAL and is based on the standard for modelling real-time systems [Rajeev Alur and David L. Dill. A theory of timed automata. Theoretical Computer
Science, 126(2):183?235, 1994.].
Times is extended by real-time tasks, checking the reachability and schedulability of a modelled automata. It too can simulate models and validate dynamic behaviour of a system: Users can see how tasks executes according to time. The simulator shows a graphical representation of the generated trace, showing the time points when the tasks are released, invoked, suspended, resumed, and completed.

TIMES can be used for code-generate, as illustrated Tobias Amnell et al [Tobias Amnell, Elena Fersman, Paul Pettersson, Hongyan Sun, and Wang Yi. Code
synthesis for timed automata. Nordic Journal of Computing, 2003.]. They generate C-code for a Lego Mindstorms system using TIMES, checking for reachability and schedulability. Simulor to what we trying to do with ECDAR. 



There are certain differences between TIMES and ECDAR, despite the missing feature of code generation in ECDAR. In contrast to TIMES's programmable logic controllers, a scheduling approach is not suf?cient to deal with unpredictable behaviour in ECDAR.
Looking at what are taking as input by the controller - TIMES simple takes a model, checking reachability and schedulability, whereas ECDAR synthesise each component in a model.



\paragraph{Composable Code Generation for Model-Based Development}
by Kirk Schloegel et al. present a framework for generating
code\cite{composable-code-generation}. They emphasize how utilizing their
framework, code generators aren't programs separated from a corresponding
graphical model as it often have been in the past. Our code generator isn't
based on this framework, however their approach on developing code generators
with focus on graphical models is related to our approach with ECDAR.

\paragraph{Code Synthesis for Timed Automata}
by Tobias Amnell et al. present a framework for the development of real-time
embedded systems\cite{Amnell:2002:CST:779110.779112}. Their work is similar to
our project. In the article the illustrate how their framework is based on timed
automata and real-time tasks - relative to our concept with ECDAR.

