In the preceding years a new level of abstraction in development have been
evolving. Utilizing a higher level of abstraction, than high-level programming
languages, we have Model Driven Development. This new paradigm is combining a
focus on automation and code generation, to enable a new way of black boxing
solutions within a multitude of specialists outside traditional programming
while securing platform independency.

\subsection {Real Time Computing}
\label{introduction-rts}

Real-time computing is the study of hardware and software systems that must
satisfy explicit response-time constraints or risk severe consequences,
including failure. Timed systems are used in a wide range of domains including
communications, embedded systems, real-time and automated control.  They can be
easily found in our environment; one of simple examples is the airbag system in
a car.  The real-time constraint in this system is the reaction time between
crash sensors receiving input and the deployment of airbags. Among some of the
important characteristics of real-time systems we can distinguish extreme
reliability and safety as they are very often safety-critical.

\subsection{ECDAR}
\label{introduction-ecdar}

The ``Environment for Compositional Design and Analysis of Real Time Systems''
(ECDAR) - is a graphical tool based on UPPAAL TIGA
\cite{behrmann_uppaal-tiga:_2006} that allows to visually create models of
real-time systems. Unlike UPPAAL \cite{larsen_uppaal_1997}, it is implementing a
complete specification theory for real time systems
\cite{David:2010:TIA:1755952.1755967,conf/atva/DavidLLNW10}. In ECDAR,
components of the system are described as automatons extended with clocks (timed
automata), that can be combined to form larger comprehensive system
descriptions. Correct specification of composition is supported by well defined
compositional reasoning theory, consisting of operators like: parallel
composition, conjunction, satisfaction checking and refinement. On the top of
that, the tool allows for scalable verification of models by querying the
implementation with verification questions \cite{conf/atva/DavidLLNW10}.


\subsection{Project}
\label{introduction-problemfield}
This paper will follow an implementation of code generation from ECDAR to
Java. The proposed implementation of the ECDAR code generator is split up in two
parts. The first part is a framework of abstract classes, implementing in as
much detail as possible the single parts of ECDAR specifications (i.e. edges,
locations, TIOA). The second is the actual code generation. Our code generator
generates sources which inherit from the abstract framework to minimize the
amount of code that needs actually to be generated. The paper will also detail
the different testing issues and benefits of implementing code generation for
TIOA, and reflect upon how this could be implemented concretely for Real Time
Systems.  However, to write programs that meet the demands of real-time systems 
in Java the Real-Time Specification for Java (RTSJ) is needed. RTSJ is a set of interfaces 
and behavioral specifications that allow for real-time programming in the Java programming language. 
Real-time in the RTSJ context means "the ability to reliably and predictably respond to a real-world event". 
So "real-time" is about predictable timing. In our implementation we don't utilize RTSJ, so we can't for instance 
guarantee thread prioritization and timing. RTSJ is only commercially available but can be integrated in to our framework in the future.